\documentclass{article}

\usepackage{amsmath}

\newcommand\ap{\vec{a}^\prime}
\newcommand\app{\vec{a}^{\prime\prime}}
\newcommand\bp{\vec{b}^\prime}
\newcommand\bpp{\vec{b}^{\prime\prime}}
\newcommand\cp{\vec{c}^\prime}
\newcommand\cpp{\vec{c}^{\prime\prime}}

\begin{document}

\title{Interpolation scheme for lattice vectors}
\author{Adam J. Jackson}
\date{}

\maketitle

As part of the application of the $\Delta V_{xc}$ method, it is necessary to generate a unit cell of target volume.
While the parameters may be optimised with an \emph{ab initio} calculation, the goal of $\Delta V_{xc}$ is to reduce computational cost and hence we should use the available information to minimise this.
Therefore we suggest linear interpolation between the surrounding minimised cells on the precalculated E-V curve.
The unit cell matrix contains a set of lattice vectors:
\begin{equation}
\mathbf{R} = \left[ \begin{array}{c} \vec{a} \\ \vec{b} \\ \vec{c}\end{array} \right]
 = \left[ \begin{array}{ccc} a_x & a_y & a_z \\ b_x & b_y & b_z \\ c_x & c_y & c_z \end{array} \right]
\end{equation}
If we are to interpolate between two precalculated lattices $\mathbf{R}^\prime$ and $\mathbf{R}^{\prime\prime}$,
then we label their lattice vectors $\ap, \bp, \cp$ and $\app, \bpp, \cpp$. The interpolated lattice $\mathbf{R} = \mathbf{R}^\prime + \lambda(\mathbf{R}^{\prime\prime} - \mathbf{R}^\prime)$, where $\lambda$ is the shared scaling parameter for interpolation. 
When lambda has been determined, the same process might be applied to atomic positions for an improved estimate.

The challenge then becomes to determine $\lambda = f(V,\mathbf{R}^\prime, \mathbf{R}^{\prime\prime})$.
The volume of a lattice unit cell,

\begin{align}
V &= \left| \vec{a} \cdot \left( \vec{b} \times \vec{c} \right) \right|. \\
\intertext{For the interpolated cell, therefore:}
V &= \left| \left( (1-\lambda)\ap + \lambda \app \right) \cdot \left[ \left( (1-\lambda) \bp + \lambda \bpp \right) \times \left( (1-\lambda) \cp + \lambda \cpp \right) \right] \right| \\
V &= \left| \left( (1-\lambda)\ap + \lambda \app \right) \cdot
    \left[\begin{array}{l}
      (1-\lambda)\bp \times (1-\lambda)\cp + (1-\lambda) \bp \times \lambda \cpp \\
      + \quad \lambda \bpp \times (1-\lambda)\cp + \lambda \bpp \times \lambda \cpp
    \end{array}\right]  \right| \\
V &= \left| \left( (1-\lambda)\ap + \lambda \app \right) \cdot
    \left[\begin{array}{l}
      (1-\lambda)^2 \bp \times \cp + \lambda(1-\lambda) \bp \times \cpp \\
      + \quad \lambda(1-\lambda) \bpp \times \cp + \lambda^2 \bpp \times \cpp
    \end{array}\right] \right| \\
V &= \left| \begin{array}{l} (1-\lambda)\ap \cdot
    \left[\begin{array}{l}
      (1-\lambda)^2 \bp \times \cp + \lambda(1-\lambda) \bp \times \cpp \\
      + \quad \lambda(1-\lambda) \bpp \times \cp + \lambda^2 \bpp \times \cpp
    \end{array}\right] \\
 \quad + \quad \lambda \app \cdot
    \left[\begin{array}{l}
      (1-\lambda)^2 \bp \times \cp + \lambda(1-\lambda) \bp \times \cpp \\
      + \quad \lambda(1-\lambda) \bpp \times \cp + \lambda^2 \bpp \times \cpp
    \end{array}\right] \end{array}\right| \\
V &= \left| \left[ \begin{array}{c}
(1-\lambda)^3 \\ \lambda(1-\lambda)^2 \\ \lambda(1-\lambda)^2 \\
\lambda^2(1-\lambda) \\ \lambda(1-\lambda)^2 \\ \lambda^2(1-\lambda) \\
\lambda^2(1-\lambda) \\ \lambda^3
\end{array} \right] \cdot
%
\left[ \begin{array}{c}
\ap \cdot \bp \times \cp \\
\ap \cdot \bp \times \cpp \\
\ap \cdot \bpp \times \cp \\
\ap \cdot \bpp \times \cpp \\
\app \cdot \bp \times \cp \\
\app \cdot \bp \times \cpp \\
\app \cdot \bpp \times \cp \\
\app \cdot \bpp \times \cpp
\end{array} \right]
%
\right| 
\end{align}
%
If we denote the right-hand vector containing all the volume permutations as $\vec{x}$,
we can express this as a simple (albeit cumbersome) cubic equation.
\begin{align}
V &= \left| \begin{array}{l}  x_1(1 - \lambda)^3 + (x_2 + x_3 + x_5) \lambda(1-\lambda)^2 \\
\quad + (x_4 + x_6 + x_7)\lambda^2(1-\lambda) + x_8 \lambda^3 \end{array}\right|\\
V &= \left| \begin{array}{l}
x_1(-\lambda^3 + 3\lambda^2 - 3\lambda + 1) \\
 + \quad (x_2 + x_3 + x_5)(\lambda^3 - 2\lambda^2 + \lambda) \\
+ \quad (x_4 + x_6 + x_7)(\lambda^2 - \lambda^3)
\quad + x_8(\lambda^3)
\end{array}\right| \\
V&= \left| \begin{array}{l}
(-x_1+x_2+x_3-x_4+x_5-x_6-x_7+x_8)\lambda^3 \\
+\quad (3x_1 - 2x_2 - 2x_3 + x_4 -2x_5 + x_6 + x_7) \lambda^2\\
+\quad (-3x_1 + x_2 + x_3 + x_5) \lambda \\
+\quad (x_1)\cdot 1
\end{array}\right|
\end{align}
This equation is solved by standard numerical methods to provide the suggested lattice vectors. A real value of $\lambda$ is required for a physically-meaningful solution, and a positive value is preferred.
\end{document}
